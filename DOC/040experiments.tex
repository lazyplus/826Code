\subsection{Datasets}
\subsubsection{DBLP dataset and other citation networks}
\subsubsection*{Description of Dataset}
DBLP is a paper dataset consists of 14,376 papers, 14,475 authors, 20 conferences. The papers included are mainly about computer science, specially in the following 4 areas: Databases, Artificial Intelligence, Data Mining and Information Retrieval. 
\subsubsection*{Methods of Experiments on Dataset}
We can leverage the co-author relationship in the datasets. In the graph, each node is a paper. If two nodes shares a same author, then we add an edge between those two nodes. Those edges are undirected, which fits perfectly with assumption of BP.
\subsubsection*{Other Citation Networks}
After further inspecting into the datasets, we found that they are not suitable for BP. One major reason is that, BP assume the edge to be undirected, while all the other citation networks, especially whose with little meta data, don't have undirected relationship as the co-author relationship in DBLP. So only DBLP is chosen.

\subsubsection{Social Networking(KDD Cup 2012)}

\subsubsection*{Description of Dataset}
This dataset comes from Tencent Weibo, one of the largest micro-blogging website in China.
It contains many different types of information and was chosen to be the material for KDD Cup 2012.

Despite the original purpose of the competition of KDD Cup, we found something interesting from the dataset:
We can get the label of some users. There is a category system in which users are labeled with hierarchy categories. And there is a following history for each user.

\subsubsection*{Methods of Experiments on Dataset}
As the BP algorithm is defined on undirected graphs, we need to fit the dataset into a more applicable format.
The following relationship is naturally uni-directional, however, most people would follow back to his/her follower if he/she found that the particular follower is similar to himself/herself.
Thus, we can keep the users and their bi-directional relationship to get a undirected graph in which edges indicates the similarity of two nodes.

After filtering the dataset, we got 2,087,070 edges in between 463,605 nodes. 6,095 of the nodes are labeled as one of 6 main categories.



\subsubsection{Product co-purchasing networks}

\subsubsection*{Description of Dataset}
The Amazon product co-purchasing dataset represents products on Amazon website. The edges represent co-purchasing relationship and all nodes are labeled in four categories: book, music, DVD and videos. There are more meta-information of items such as subjects of the books and ratings from customers.

\subsubsection*{Methods of Experiments on Dataset}

Co-purchased products can be similar, i.e. belong to the same category.
So, we can infer the categories for all products in the co-purchasing network with several seeds in the whole network.
An even more sophisticated experiment would involve with the subject of items. For example, we can infer the subjects of all the books.

\subsubsection{Flickr photo network}

\subsubsection*{Description of Dataset}
This data set is from MIR (Multimedia Information Retrieval) FLICKR Retrieval Evaluation.In this dataset, photos are associated with one or more tags from Flickr user. Photos are also annotated manually by annotator with a category. We build this network of photos based on whether two photos share same tags. 

There are 22,872 photos that have at least one tag in this dataset. Tags are from Flickr users, like ‘dog’, ‘doors’, ‘ocean’ etc. For photos with at least one tag, each of them is associated with 9.8 tags on average. The photo with largest number of tags has 75 tags. The photo with smallest number of tags has only 1 tag. The number of photos only with 1 tag is 813. The number of photos with 2 tags is 987.

The photos are manually annotated with one or more of the 24 categories like ‘animals’, ‘baby’, ‘indoor’ etc. There are 24,581 photos with at least one category. For photos with at least one category, each of them is associated with 3.8 categories on average.  The most widely used category is ‘people’ (10,373 times) followed by ‘structures’ (9992 times). 

\subsubsection*{Methods of Experiments on Dataset}
We can build a network of photos based on whether two photos share tags. If we build a network of photos based on whether two photos share 1 tag, the average degree is 781. If we build a network of photos based on whether 2 photos share at least two tags, the average degree is 92. If we build a network of photos based on whether two photos share at least 3 tags, the average degree is 16. In later experiment, we can try different network building criterion.

With this network of photos, we can run BP and \textbf{FaBP}, and predict the category of photos.


 \subsection{Preliminary Experiments}
We did some preliminary experiments on BP and FastBP algorithms.

By extracting all the items, whose groups are DVD, Video and Music from the Amazon co-purchase dataset, we get 149,102 nodes and 253,407 edges connecting them.
The classification accuracy of BP is around 72\% with 5\% nodes labeled.

We then reduced the dataset to just two types (DVD and Video) of items and run FastBP on it.
The result shows that there are only 4,274 nodes have non-zero bias from 50\% probability to either type (2,298 (5\%)of them have prior value) after 10 iterations.
